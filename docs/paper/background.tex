There are a huge number of fuzzing tools\cite{afl,aflosx,winafl,peach22,syzkaller,ossfuzz,driller,radamsa,ni,zzuf,synfuzz,brundlefuzz,honggfuzz,kafl,vuzzer,boofuzz} which are publicly available, many of
which are very useful on real-world binaries.
However, each tool
typically only handles one very specific use case or contains other real-world limitations, and most are not designed to scale
out.  For instance, there are a number of system call kernel fuzzers\cite{syzkaller,trinity,kafl,osxfuzz}, different ones for
fuzzing \IOCTLs{}\cite{ioctlfuzzer,ioctlbf}, some userland Linux fuzzers which only effectively work with
command line programs which were written in C or C++.  Some only function with
source code, only work on 32-bit Linux\cite{vuzzer}, or require users to manually specify
the data format which the target is
expecting.\cite{peach,boofuzz}  Finally, there is a category of tools which work amazingly well
on toy programs but do not work on production software due to bugs or lack of
support for features such as multithreading.\cite{grimmdriller,angrissues}

Lack of compatibility with production software is typically not viewed as an issue in academic work, as the problem to address can be
scoped based on the tools that are available and one may assume that the problem can
be solved in other situations, but leave the proof for future work.  The researchers
are typically correct in their assumption, however practitioners need tools
that work in practice, not theoretical solutions.

In industry, the situation is dictated by the target software,
and there is often not a choice as to what language it is
written in, whether source code is available, what operating system it runs
on, or which CPU architectures it supports. This leaves the security
professional to choose between the limited set of tools that can handle the specific requirements of their target, many of which will
turn out to be mutually exclusive. This lack of available tools also creates a new problem, as a common
response to this dilemma is to put together a custom tool which meets their
needs, and to do so in the shortest amount of time possible.  Furthermore,
this results in the same code being re-written for different platforms, or
sometimes for the same platform simply because the security professional was
unaware of existing implementations.  The adapted tool will probably
contain some of the same bugs and limitations that were in the initial
version of the existing tool, which may or may not get fixed before it is
abandoned.

In short, while that state of security research is advancing rapidly, the
tools to bring their benefits to life are sorely lacking.  Though there are some
fuzzing projects which come close, such as OSS-Fuzz,\cite{ossfuzz} there
are not any fuzzing tools which are freely available, work on closed source applications,
can be run in a distributed manner, and run against Windows, Linux, and macOS applications.
