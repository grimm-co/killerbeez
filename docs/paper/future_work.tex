% Summary of the future work
The short term focus will be to pull in existing technologies from other
projects and get them integrated with Killerbeez and running on all the
supported platforms.  There is ample research, tools and techniques available
now which have yet to be applied in different domains. Once the state of the
art has been incorporated, more automation and designing interfaces to pull in
projects which don't fit into the existing modules will be the next priority.


% More Instrumentation modules: PIN, Dyninst, IPT on Windows
% More Drivers: monitor dialog boxes
%               more input methods (tie in existing kernel fuzzers)
% More Mutators: Python interface to pull in BrundleFuzz and 010 mutators
% Pull in concepts from academic papers which modified AFL (see citations for a list)
%   Mention that we will prioritize papers which released code, and things like
%   Angora which promised to release code then never did, will be last in line.
% Integrate How to generate seeds
%   Give shout out to JRozner with his project & DEF CON talk \cite{synfuzz}


% The picker and IPT instrumentation could be updated to be compatible with
% one another.  This would be a decent amount of work, for a questionable
% amount of benefit.


% Not sure if we want to take on the things below or not, but if so they will
% be in the far future.
% Unsolved problems:

% How to choose targets
% Where to get seeds?
%   Search the web, very ad-hoc and manual process
% Avoiding the easy-to-find crashes to get to the more interesting ones
%   Could be done with smarter input generators/mutators & automated static/dynamic analysis
% Targets which include checksums, compression or encryption
%   Typical solution, modify the source/executable to remove those checks


% Room for improvement:

% Detecting non-crashing errors (especially without source code for the target)


