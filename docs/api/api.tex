\documentclass{article}
% 1 = definition, 2 = description, 3 = args/return value
\def\api#1#2#3{
\bigskip
% Can't make \texttt bold...
% Maybe use \lstinputlisting ?
\texttt{#1}
\par
#2
\par
\begin{itemize}
#3
\end{itemize}
}

\def\killerbeez{Killerbeez}
\def\apiVersion{0.1}
\def\apiDate{2018.07.30}

\usepackage{textcomp}
\usepackage{listings}
\lstset{basicstyle=\ttfamily\small,showstringspaces=false,upquote=true}
\usepackage{graphicx}

\makeatletter         
\def\@maketitle{
    \begin{center}
        {\Huge \bfseries \sffamily \@title }\\[4ex] 
        {\Large \@date}\\[8ex]
        \includegraphics{killerbeez-logo.png}
    \end{center}
}
\makeatother



\title{\killerbeez{} API}
\author{GRIMM}
\date{\apiDate{}}

\begin{document}
% Nice cover page
\thispagestyle{empty}
\maketitle
\newpage

% Table of Contents
\tableofcontents
\newpage

\section{Overview}
This document will cover the version \apiVersion{} API for each module, along
with a quick high-level summary of what it does.

\par
The APIs are all specified in C, as this provides a consistent language and is
explicit about data types which means there's no need for a separate Python
specification.  The C code is frequently wrapped with Python (via ctypes), but
modules are typically written in C code as they run considerably faster when
it is all native code.

\section{Manager}
The manager is what coordinates a fuzz job.  It decides which seeds to use,
which mutators to run and sends this information to the client, which kicks
off one or more Main Fuzzers.  The client will then also handle getting the
results back to the manager.

\section{Fuzzer}
This will run many iterations of a single seed and a single mutator against a
target program.  For efficiency, this will be run on the same computer (which
means same O/S) as the target binary.  This component will be an executable that
the manager executes on each of the target systems.  The arguments for this
function are defined in the usage function of the fuzzer.

\section{Mutator}
\label{mutator}
The mutator modules are what actual mutate the seed files.  These would include
things like a bit flipper, byte munger and so forth.  They are given a seed file
and optionally some state information.  The state information is module-specific
and allow the mutator to pick up where it left off.  For example, the bit
flipper mutator module, which simply flips one bit in the input file, would just
need to record what bit to flip as their state.  On the other hand, the byte
munger will need to keep track of which byte its modifying and what the new
value is for that byte (e.g.  stomp the fourth byte with 0x41).  Additionally,
each mutator will have a variety of configuration options that can be specified
that will be mutator specific.  Both the mutator state and options will be
specified as a JSON char arrays.

\par
Anything which is mutator specific will only be used within the mutator
functions.  All other components will treat these items as opaque strings/blobs.


\api{void init(mutator\_t * m)
}{
This function fills in m with all of the function pointers for this mutator.
% TODO Put this note in a sidebar
Note: This function only appears when compiled as a module.  When
ALL\_MUTATORS\_IN\_ONE is defined, this function will not exist, as there would
be a name collision with all the other init() functions from other modules and
there will not be any need for obtaining this struct as all the functions will
just be called directly.  Its just the code which uses modules which will want
to use this struct.
}{
\item m - a pointer to a mutator\_t structure that will be filled in with the
function pointers that define this mutator.
\item return value - none
}


\api{void * create(char * options, char * state, char * input,
size\_t input\_length)
}{
This function will allocate and initialize the mutator structure.  The lifetime
of the structure which is allocated will be until the cleanup() function is
called.
}{
\item options - a json string that contains the mutator specific string of
options.
\item state - used to load a previously dumped state (with the get\_state()
function), that defines the current iteration of the mutator.  This will be a
mutator specific JSON string.  Alternatively, NULL can be provided to start a
mutator without a previously dumped state.
\item input - used to produce new mutated inputs later when the mutate function
is called
\item input\_length - the size of the input buffer
\item return value - a mutator specific structure or NULL on failure.  The
returned value should not be used for anything other than passing to the various
Mutator API functions.
}


\api{void cleanup(void * mutator\_state)
}{
This function will release any resources that the mutator has open and free the
mutator state structure.
}{
\item mutator\_state - a mutator specific structure previously created by the
create function.  This structure will be freed and should not be referenced
afterwards.
}


\api{
int mutate(void * mutator\_state, char * output\_buffer, size\_t buffer\_length)
}{
This function will mutate the input given in the create function and return it
in the output\_buffer argument.  The size of the output\_buffer will be mutator
specific.  For example, some mutators may require this buffer to be larger than
the original input (passed to the create() function) as its going to extend the
original input in some way.  Other mutators will want it to be the same size.
Guidance on this will be specified by the mutator specific documentation.
}{
\item mutator\_state - a mutator specific structure previously created by the
create function.
\item output\_buffer - a buffer that the mutated input will be written to
\item buffer\_length - the size of the passed in buffer argument
\item return value - the length of the mutated data, 0 when the mutator is out
of mutations, or -1 on error
}


\api{char * get\_state(void * mutator\_state)
}{
This function will return the state of the mutator.  The returned value can be
used to restart the mutator at a later time, by passing it to the create or
set\_state function.  It is the callers responsibility to free the memory
allocated here.
}{
\item mutator\_state - a mutator specific structure previously created by the
create function.
\item return value - a buffer that defines the current state of the mutator.
This will be a mutator specific JSON string.
}


\api{void free\_state(char * state)}{
This function will free a previously dumped state (via the get\_state function)
of the mutator.
}{
\item state - a previously dumped state buffer obtained by the get\_state
function.
}


\api{int set\_state(void * mutator\_state, char * state)
}{
This function will set the current state of the mutator.  This can be used to
restart a mutator once from a previous run.
}{
\item mutator\_state - a mutator specific structure previously created by the
create function.
\item state - a previously dumped state buffer obtained by the get\_state
function.  This will be
a mutator specific JSON string.
\item return value - 0 on success or non-zero on failure
}


\api{
int get\_current\_iteration(void * mutator\_state)
}{
This function will return the current iteration count of the mutator, i.e. how
many mutations have been generated with it.
}{
\item mutator\_state - a mutator specific structure previously created by the
create function.
\item return value - the number of previously generated mutations
}


\api{int get\_total\_iteration\_count(void * mutator\_state)
}{
This function will return the total possible number of mutations with this
mutator.  For some mutators, this value wont be possible to predict or the
mutator will be capable of an infinite number of mutations.
}{
\item mutator\_state - a mutator specific structure previously created by the
create function.
\item return value - the number of possible mutations with this mutator.  If
this number cant be predicted or is infinite, -1 will be returned.
}


\api{int set\_input(void * mutator\_state, char * new\_input,
size\_t input\_length)
}{
This function will set the input (saved in the mutators state) to something new.
This can be used to reinitialize a mutator with new data, without reallocating
the entire state struct.
}{
\item mutator\_state - a mutator specific structure previously created by the
create function.
\item new\_input - The new input used to produce new mutated inputs later when
the mutate function is called
\item input\_length - the size in bytes of the input buffer.
\item return value - 0 on success and -1 on failure
}


\api{int help(char ** help\_str)
}{
This function sets a help message for the mutator. This is useful if the mutator
takes a JSON options string in the create() function.
}{
\item help\_str - A double pointer that will be updated to point to the new help
string.
\item return value - 0 on success and -1 on failure
}


\section{Driver}
\label{driver}
The driver will be the component that runs the program being fuzzed.  The driver
should start the program, feed in the input, and determine when the program is
done processing the input.  This component may need to be customized per
target application.

\par
Anything which is driver specific will only be used within the driver functions.
All other components will treat these items as opaque strings/blobs.


\api{void * create(char * options, instrumentation\_t * instrumentation,
void * instrumentation\_state, mutator\_t * mutator, void * mutator\_state)
}{
This function will allocate and initialize the driver structures.  If the driver
is going to be testing a long-running process, this function should start that
process.  Anything that needs to be done before a fuzzing run can start should
be done here.
}{
\item options - a JSON string that contains the driver specific string of
options.
\item instrumentation - a pointer to an instrumentation instance that the driver
will use to instrument the requested program.  The caller should initialize this
instrumentation instance before the create call to the driver, and then free it
after cleaning up the driver.  This parameter is optional and can be set to NULL
if the caller does not wish to use an instrumentation with the driver.
\item instrumentation\_state - a pointer to the instrumentation state for the
passed in instrumentation.  This parameter is optional and can be set to NULL
if the caller does not wish to use an instrumentation with the driver.
\item mutator - a pointer to a mutator instance that the driver can use to
obtain the next input (for use in the \texttt{test\_next\_input} function).
This parameter is optional and can be set to NULL if the caller does not wish to
use a mutator with the driver.  Without this parameter, the
\texttt{test\_next\_input} and \texttt{get\_last\_input} functions will be
unavailable.
\item mutator\_state - a pointer to the mutator state for the passed in mutator.
This parameter is optional and can be set to NULL if the caller does not wish to
use a mutator with the driver.
\item return value - a driver specific structure or NULL on failure.  The
returned value should not be used for anything other than passing to the various
Driver API functions.
}

\api{void cleanup(void * driver\_state)
}{
This function will kill any processes created by the driver and clean up
anything else that was created to help fuzzing.  It will also free the driver
state.
}{
\item driver\_state - a driver specific structure previously created by the
create function.  This structure will be freed and should not be referenced
afterwards.
}

\api{int test\_input(void * driver\_state, char * buffer, size\_t length)
}{
This function will cause the program being fuzzed to be tested against the given
input.  This function should block execution until the program being fuzzed has
finished processing the given input.
}{
\item driver\_state - a driver specific structure previously created by the
create function.
\item buffer -  the input that should be tested
\item length - the length of the buffer argument
\item return value - 0 on success, -1 on failure
}

\api{int test\_next\_input(void * driver\_state);
}{
This function uses the mutator given during the driver creation to retrieve the
next mutated input and test it against the target program.  This function blocks
execution until the program being fuzzed has finished processing the mutated
input. It will report whether the fuzzed process crashed or hung, or neither.
This function is only available if a mutator was given to the driver in the
\texttt{create} function.
}{
\item driver\_state - a driver specific structure previously created by the
create function.
\item return value - 0 on success, -1 on failure, or -2 if the mutator has run
out of inputs to to mutate.
}

\api{void * get\_last\_input(void * driver\_state, int * length);
}{
This function retrieves the most recent mutated input that was tested with the
\texttt{test\_next\_input} function.  This function is only available if a
mutator was given to the driver in the \texttt{create} function.
}{
\item driver\_state - a driver specific structure previously created by the
create function.
\item length - a pointer to an integer that will be set to the length of the
returned buffer.
\item return value - on success this function will return a buffer containing
the last input that was tested, or NULL on failure.  This pointer should not be
freed by the caller, and is only valid until the next call to
\texttt{test\_next\_input}.
}


\api{int help(char ** help\_str)
}{
This function sets a help message for the driver. This is useful if the driver
takes a JSON options string in the create() function.
}{
\item help\_str - A double pointer that will be updated to point to the new help
string.
\item return value - 0 on success and -1 on failure
}


\section{Instrumentation}
\label{instrumentation}
The instrumentation modules are what track the state of a process and
determine if a path through the process is new.  This will include things such
as QEMU (for Linux), LLVM (for source), PIN, Dynamo-RIO, Dyninst, and Intel PT.
They are optionally given some state information.  The state information is
module-specific and is used to tell the instrumentation module which paths have
been previously hit.  Additionally, each instrumentation module will have a
variety of configuration options that can be specified that will be
specific to that instrumentation module.  These options will be specified as a
JSON char array.

\par
Anything which is instrumentation specific will only be used within the
instrumentation functions.  All other components will treat these items as
opaque strings/blobs.

\api{void * create(char * options, char * state)
}{
This function will create and return an instrumentation struct that defines
the instrumentation's state.  The state argument will be used to load the
previously executed paths through the fuzzed program.
}{
\item options - a JSON string that contains the instrumentation specific options
\item state - used to load a previously dumped state (produced by the get\_state()
function), that defines the current paths seen by the instrumentation.
Alternatively, NULL can be provided to start an instrumentation without a
previously dumped state
\item return value - an instrumentation specific structure or NULL on failure.
The returned value should not be used for anything other than passing to the
various Instrumentation API functions
}

\api{void cleanup(void * instrumentation\_state)
}{
This function will release any resources that the instrumentation has open and
free the instrumentation state.
}{
\item instrumentation\_state - an instrumentation specific structure previously
created by the create function.  This structure will be freed and should not be
referenced afterwards
}

\api{char * get\_state(void * instrumentation\_state, int *out\_length)
}{
This function will return the state information holding the previous execution
path info.  The returned value can later be passed to the instrumentation
create() function to load the state back into an instrumentation struct.
It is the caller's responsibility to free the memory allocated and returned
here using the free\_state() function.
}{
\item instrumentation\_state - an instrumentation specific structure previously
created by the create function
\item out\_length - this pointer will be filled with the length of the returned
state buffer
\item return value - a buffer that holds information about the previous
execution paths as a JSON char array.
}

\api{void free\_state(char * state)
}{
This function will free a previously dumped state (via the get\_state()
function) of the instrumentation.
}{
\item state - a previously dumped state buffer obtained by the get\_state()
function
}

\api{int set\_state(void * instrumentation\_state, char * state)
}{
This function will set the previous execution paths of the instrumentation.
This can be used to restart an instrumentation once it has been created.
}{
\item instrumentation\_state - an instrumentation specific structure previously
created by the create() function
\item state - a previously dumped state buffer obtained by the get\_state()
function
\item return value - 0 on success or non-zero on failure
}

\api{void * merge(void * instrumentation\_state,
void * other\_instrumentation\_state)
}{
This function will merge two sets of instrumentation coverage data.  The
resulting instrumentation state will include the tracked coverage from both
instrumentation states.  Both instrumentation states must have the same
instrumentation options (what to track coverage of, which modules, etc.)
specified, and generally need to be produced by the same instrumentation
module in order for the merge to work correctly.  It's possible that two
different instrumentation modules may produce state information in the same
format, however this is up to them and not something guaranteed by this
specification.  Neither argument will be modified nor freed.  It is the caller's
responsibility to free the memory allocated and returned here using the
free\_state() function.
}{
\item instrumentation\_state - an instrumentation specific structure previously
created by the create() function
\item other\_instrumentation\_state - a second instrumentation specific structure
previously created by the create() function that should be merged with the first
\item return value - an instrumentation specific structure that combines the
coverage information from both of the instrumentation states or NULL on failure
}

\api{int enable(void * instrumentation\_state, HANDLE * process, char * cmd\_line,
char * input, size\_t input\_length)
}{
This function will enable the instrumentation module for a specific process and
runs that process.  If the process needs to be restarted, it will be.
}{
\item instrumentation\_state - an instrumentation specific structure previously
created by the create() function
\item process - a pointer to a handle for the process on which the
instrumentation was enabled
\item cmd\_line - the command line of the fuzzed process on which to enable
instrumentation
\item input - pointer to the buffer containing the input data that should be
sent to the fuzzed process
\item input\_length - the length of the input parameter
\item return value - 0 on success, non-zero on failure
}

\api{int is\_new\_path(void * instrumentation\_state)
}{
This function will determine whether the process being instrumented has taken a
new path.  It should be called after the process has finished processing the
tested input. If \texttt{is\_new\_path} is called prior to \texttt{enable}, it
will return failure, as the fuzzing of processes has not been started yet.
}{
\item instrumentation\_state - an instrumentation specific structure previously
created by the create() function
\item return value - 1 if the previously setup process (via the enable()
function) took a new path, 0 if it did not, or -1 on failure
}

\api{int get\_fuzz\_result(void * instrumentation\_state)
}{
This function will return the result of the fuzz job. It should be called
after the process has finished processing the tested input, i.e. after a
successful \texttt{is\_process\_done}. If \texttt{get\_fuzz\_result} is called
prior to \texttt{enable}, it will return failure, as the fuzzing of processes
has not been started yet.
}{
\item instrumentation\_state - an instrumentation specific structure previously
created by the create() function
\item return value - either \texttt{FUZZ\_NONE}, \texttt{FUZZ\_HANG},
\texttt{FUZZ\_CRASH}, or \texttt{FUZZ\_ERROR} on error.
}

\api{int get\_module\_info(void * instrumentation\_state, int index, int * is\_new,
char ** module\_name, char **info, int size)
}{
This function is optional and not required for the fuzzer to work.  It
can be used to obtain coverage information for each executable/library
separately.  This function returns information about each of the separate
modules (shared libraries such as .dll, .so, .dynlib). If
\texttt{get\_module\_info} is called and requests the \texttt{is\_new} or
\texttt{info} parameters prior to \texttt{enable} being called, it will return
failure, as the fuzzing of processes has not been started yet.
}{
\item instrumentation\_state - an instrumentation specific structure previously
created by the create() function
\item index - an index into the module list for the module about which
information should be retrieved.  The return value will indicate if a module
exists for this index.  Indices start at 0 and increase from there
\item is\_new - This parameter returns whether or not the last run of the
instrumentation returned a new path for the module with the specified index.  In
order for the information returned in this parameter to be accurate, the
is\_new\_path method should be called first.  This parameter is optional and can
be set to NULL, if you do not want this information
\item module\_name - This parameter returns the filename of the module at the
specified index.  This parameter is optional and can be set to NULL, if you do
not want this information.  This parameter should not be freed by the caller
\item info - This parameter returns the per-instrumentation path info for the
module with the specified index.  For example, for the DynamoRIO module, the
returned info is an AFL style bitmap of the edges.  This parameter is optional
and can be set to NULL, if you do not want this information.  This parameter
should not be freed by the caller
\item size - This parameter returns the size of the per-instrumentation path
info in the returned info parameter.  This parameter is optional and can be set
to NULL, if you do not want this information
\item return value - non-zero if module with the specified index cannot be
found, or 0 if it is found
}

\api{instrumentation\_edges\_t * get\_edges(void * instrumentation\_state,
int index)
}{
This function is optional and not required for the fuzzer to work.  It is used
by the tracer.  This function returns an array of basic block edges that
occurred in the most recent run of the instrumentation. If \texttt{get\_edges}
is called prior to \texttt{enable}, it will return failure, as the fuzzing of
processes has not been started yet.
}{
\item instrumentation\_state - an instrumentation specific structure previously
created by the create() function.
\item index - If per-module instrumentation information is enabled, this
parameter is an index into the module list for the module about which edges
should be retrieved.  The return value will indicate if a module exists for this
index.  Indices start at 0 and increase from there.  If per-module
instrumentation information is NOT enabled, then this parameter is ignored and
the general edges array will be returned.
\item return value - NULL if an array of basic block edges was not tracked for
the most recent instrumentation run or per-module instrumentation is enabled and
the requested index was not found.  Otherwise, an instrumentation\_edges\_t pointer
that contains an array of basic block edges that were hit in the most recent
instrumentation run.  The returned pointer should be freed by the caller.
}

\api{int is\_process\_done(void * instrumentation\_state)
}{
This function is used by the driver to determine if an instrumented process has
finished being fuzzed. If the process is done being fuzzed, then
\texttt{get\_fuzz\_result} is ready to be called. If \texttt{is\_process\_done}
is called prior to \texttt{enable}, it will return failure, as the fuzzing of
processes has not been started yet.
}{
\item instrumentation\_state - an instrumentation specific structure previously
created by the create() function.
\item return value - 0 if the process has finished being fuzzed, 1 if not, -1
on error.
}


\api{int help(char ** help\_str)
}{
This function sets a help message for the instrumentation. This is useful if the
instrumentation takes a JSON options string in the create() function.
}{
\item help\_str - A double pointer that will be updated to point to the new help
string.
\item return value - 0 on success and -1 on failure
}


\section{Structures}
\label{structures}
This section describes the structures used throughout the API.  For each of the
top level components, there is a structure which defines the available functions
in that component.  This allows for a common interface among all of the
available implementations of a component.

\vbox{\lstinputlisting[
  label={lst:mutatort},
  caption={\texttt{mutator\_t} struct definition},
  language=C,
  captionpos=b
  ]{files/mutator_t.c}
}

The \texttt{mutator\_t} structure, shown in Listing \ref{lst:mutatort}, defines
all of the common interfaces for each mutator.  The definitions of each of the
function pointers in the \texttt{mutator\_t} structure is described in Section
\ref{mutator}.


\section{Helper Utilities}
\label{helpers}
In addition to the fuzzer, a few utilities have been created that help with the
fuzzing process.  This section describes these helper utilities and their role
in the \killerbeez{} architecture.

\subsection{Merger}
The merger combines multiple sets of instrumentation data into one
instrumentation state.  The resulting instrumentation state will include the
tracked coverage from all of the input instrumentation states. This allows
multiple instances of the fuzzer to share instrumentation data, and ignore paths
that the other fuzzers found.

\subsection{Picker}
The picker helps the user decide which libraries should be instrumented while
fuzzing.  This is accomplished by running the target program and recording
coverage information on each of the loaded libraries. It then analyzes the
coverage information for each library to determine which libraries the
coverage information varies based on the input file.  These libraries are most
likely the ones that process the input file, and thus the most likely targets
for fuzzing.




%\section{Database}
% TODO: document me!

%\section{Tracer}
% TODO: document me!

%\section{Input Generator}
% TODO: create me and then document me!

\end{document}

