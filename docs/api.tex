\documentclass{article}
\title{Killerbeez API}
\author{GRIMM}
\date{2017.12.22}
% 1 = definition, 2 = description, 3 = args/return value
\def\api#1#2#3{
\bigskip
% Can't make \texttt bold...
% Maybe use \lstinputlisting ?
\texttt{#1}
\par
#2
\par
\begin{itemize}
#3
\end{itemize}
}

\def\killerbeez{Killerbeez}
\def\apiVersion{0.1}
\def\apiDate{2018.07.30}


\begin{document}
% Nice cover page
\thispagestyle{empty}
\maketitle
\newpage

% Table of Contents
\tableofcontents
\newpage

\section{Overview}
This document will cover the specific API of each module, along with a quick
high-level summary of what it does.

\par
The APIs are all specified in C, as this provides a consistent language and is
explicit about data types which means there's no need for a separate Python
specification.  The C code is frequently wrapped with python (via ctypes), but
modules are often included as pure C code as they run considerably faster when
it is all native code.

\section{Manager}
The manager it what coordinates a fuzz job.  It decides which seeds to use,
which mutators to run and kicks off one or more Main Fuzzers.

\section{Main Fuzzer}
This will run many iterations of a single seed and a single mutator against a
target program.  For efficiency, this will be run on the same computer (which
means same O/S) as the target binary.  This component will be an executable that
the manager executes on each of the target systems.  The arguments for this
function are defined in the usage function of the fuzzer.

\section{Mutator}
The mutator modules are what actual mutate the seed files.  These would include
things like a bit flipper, byte munger and so forth.  They are given a seed file
and optionally some state information.  The state information is module-specific
and allow the mutator to pick up where it left off.  For example, the bit
flipper mutator module, which simply flips one bit in the input file, would just
need to record what bit to flip as their state.  On the other hand, the byte
munger will need to keep track of which byte its modifying and what the new
value is for that byte (e.g.  stomp the fourth byte with 0x41).  Additionally,
each mutator will have a variety of configuration options that can be specified
that will be mutator specific.  Both the mutator state and options will be
specified as a JSON char arrays.

\par
Anything which is mutator specific will only be used within the mutator
functions.  All other components will treat these items as opaque strings/blobs.


\api{void init(mutator\_t * m)
}{
This function fills in m with all of the function pointers for this mutator.
% TODO Put this note in a sidebar
Note: This function only appears when compiled as a module.  When
ALL\_MUTATORS\_IN\_ONE is defined, this function will not exist, as there would
be a name collision with all the other init() functions from other modules and
there will not be any need for obtaining this struct as all the functions will
just be called directly.  Its just the code which uses modules which will want
to use this struct.
}{
\item m - a pointer to a mutator\_t structure that will be filled in with the
function pointers that define this mutator.
\item return value - none
}


\api{void * create(char * options, char * state, char * input,
size\_t input\_length)
}{
This function will allocate and initialize the mutator structure.  The lifetime
of the structure which is allocated will be until the cleanup() function is
called.
}{
\item options - a json string that contains the mutator specific string of
options.
\item state - used to load a previously dumped state (with the get\_state()
function), that defines the current iteration of the mutator.  This will be a
mutator specific JSON string.  Alternatively, NULL can be provided to start a
mutator without a previously dumped state.
\item input - used to produce new mutated inputs later when the mutate function
is called
\item input\_length - the size of the input buffer
\item return value - a mutator specific structure or NULL on failure.  The
returned value should not be used for anything other than passing to the various
Mutator API functions.
}


\api{void cleanup(void * mutator\_state)
}{
This function will release any resources that the mutator has open and free the
mutator state structure.
}{
\item mutator\_state - a mutator specific structure previously created by the
create function.  This structure will be freed and should not be referenced
afterwards.
}


\api{
int mutate(void * mutator\_state, char * output\_buffer, size\_t buffer\_length)
}{
This function will mutate the input given in the create function and return it
in the output\_buffer argument.  The size of the output\_buffer will be mutator
specific.  For example, some mutators may require this buffer to be larger than
the original input (passed to the create() function) as its going to extend the
original input in some way.  Other mutators will want it to be the same size.
Guidance on this will be specified by the mutator specific documentation.
}{
\item mutator\_state - a mutator specific structure previously created by the
create function.
\item output\_buffer - a buffer that the mutated input will be written to
\item buffer\_length - the size of the passed in buffer argument
\item return value - the length of the mutated data, 0 when the mutator is out
of mutations, or -1 on error
}


\api{char * get\_state(void * mutator\_state)
}{
This function will return the state of the mutator.  The returned value can be
used to restart the mutator at a later time, by passing it to the create or
set\_state function.  It is the callers responsibility to free the memory
allocated here.
}{
\item mutator\_state - a mutator specific structure previously created by the
create function.
\item return value - a buffer that defines the current state of the mutator.
This will be a mutator specific JSON string.
}


\api{void free\_state(char * state)}{
This function will free a previously dumped state (via the get\_state function)
of the mutator.
}{
\item state - a previously dumped state buffer obtained by the get\_state
function.
}


\api{int set\_state(void * mutator\_state, char * state)
}{
This function will set the current state of the mutator.  This can be used to
restart a mutator once from a previous run.
}{
\item mutator\_state - a mutator specific structure previously created by the
create function.
\item state - a previously dumped state buffer obtained by the get\_state
function.  This will be
a mutator specific JSON string.
\item return value - 0 on success or non-zero on failure
}


\api{
int get\_current\_iteration(void * mutator\_state)
}{
This function will return the current iteration count of the mutator, i.e. how
many mutations have been generated with it.
}{
\item mutator\_state - a mutator specific structure previously created by the
create function.
\item return value - the number of previously generated mutations
}


\api{int get\_total\_iteration\_count(void * mutator\_state)
}{
This function will return the total possible number of mutations with this
mutator.  For some mutators, this value wont be possible to predict or the
mutator will be capable of an infinite number of mutations.
}{
\item mutator\_state - a mutator specific structure previously created by the
create function.
\item return value - the number of possible mutations with this mutator.  If
this number cant be predicted or is infinite, -1 will be returned.
}


\api{int set\_input(void * mutator\_state, char * new\_input,
size\_t input\_length)
}{
This function will set the input (saved in the mutators state) to something new.
This can be used to reinitialize a mutator with new data, without reallocating
the entire state struct.
}{
\item mutator\_state - a mutator specific structure previously created by the
create function.
\item new\_input - The new input used to produce new mutated inputs later when
the mutate function is called
\item input\_length - the size in bytes of the input buffer.
\item return value - 0 on success and -1 on failure
}


\api{int help(char ** help\_str)
}{
This function sets a help message for the mutator. This is useful if the mutator
takes a JSON options string in the create() function.
}{
\item help\_str - A double pointer that will be updated to point to the new help
string.
\item return value - 0 on success and -1 on failure
}


%\section{Driver}
% TODO: document me!

%\section{Instrumentation}
% TODO: document me!

%\section{Database}
% TODO: document me!

%\section{Tracer}
% TODO: document me!

%\section{Input Generator}
% TODO: create me and then document me!

\end{document}

