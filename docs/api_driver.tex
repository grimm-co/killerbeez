The driver will be the component that runs the program being fuzzed.  The driver
should start the program, feed in the input, and determine when the program is
done processing the input.  This component may need to be customized per
target application.

\par
Anything which is driver specific will only be used within the driver functions.
All other components will treat these items as opaque strings/blobs.


\api{void * create(char * options, instrumentation\_t * instrumentation,
void * instrumentation\_state)
}{
This function will allocate and initialize the driver structures.  If the driver
is going to be testing a long-running process, this function should start that
process.  Anything that needs to be done before a fuzzing run can start should
be done here.
}{
\item options - a json string that contains the driver specific string of
options.
\item instrumentation - a pointer to an instrumentation instance that the driver
will use to instrument the requested program.  The caller should initialize this
instrumentation instance before the create call to the driver, and then free it
after cleaning up the driver.
\item instrumentation\_state - a pointer to the instrumentation state for the
passed in instrumentation.
\item return value - a driver specific structure or NULL on failure.  The
returned value should not be used for anything other than passing to the various
Driver API functions.
}

\api{void cleanup(void * driver\_state)
}{
This function will kill any processes created by the driver and clean up
anything else that was created to help fuzzing.  It will also free the driver
state.
}{
\item driver\_state - a driver specific structure previously created by the
create function.  This structure will be freed and should not be referenced
afterwards.
}

\api{int test\_input(void * driver\_state, char * buffer, size\_t length)
}{
This function will cause the program being fuzzed to be tested against the given
input.  This function should block execution until the program being fuzzed has
finished processing the given input.
}{
\item driver\_state - a driver specific structure previously created by the
create function.
\item buffer -  the input that should be tested
\item length - the length of the buffer argument
\item return value - 0 on success, non-zero on failure
}
